\section*{ВВЕДЕНИЕ}

	\addcontentsline{toc}{section}{\textbf{ВВЕДЕНИЕ}}
	
	Расчётно-исследовательская работа магистранта представляет  результат-отчёт дисциплины "Интеллектуальное управление в условиях неопределенностей. Основная часть аналитических расчётов, а так же математическое моделирование выполненны в пакете программ Matlab.
	\par
	В ходе выполнения работы, необходимо:
	
	\begin{enumerate}
			\item
			Построить модель траекторной чувствительности (МТЧ) непрерывного объекта управления (НОУ). С использованием матрицы управляемости агрегированной системы, ранжировать параметр $q_{j}$ по потенциальной чувствительности к ним выхода ОУ;
	\item
		 	Построить модель траекторной чувствительности (МТЧ) дискретного объекта управления (ДОУ) к вариации интервала дискретности;
		\item
			Построить модель траекторной чувствительности (МТЧ) спроектированной непрерывной системы по каждому из полученных параметров и для значения $|\bm{\mathit{\Delta}}q_{i}|=0.3$. Выделить доминирующие параметры по степени их влияния на величину $ \sigma $ и длительность $t_{p} $ переходного процесса;
		\item
			Построить матрицу функций модальной чувствительности (МФМЧ) и выделить неблагоприятное сочетание вариаций параметров;
		\item
			Методом модального управления, базовый алгоритм которого дополняется контролем нормы $ \|F_{0}\| $ медианной составляющей интервальной матрицы $[F]  $ спроектированной системы для целей вычисления оценки $ \delta_{1}F $ ее относительной интервальности Исследовать свойство робастной устойчивости полученной системы с помощью метода В.Л. Харитонова;
    	\item
			Оценить алгебраическую реализуемость неадаптивного и адаптивного управления, обеспечивающего параметрическую инвариантность выхода системы, и синтезировать их.

\end{enumerate}
\newpage
\section*{Исходные данные для выполнения расчетной работы}

	\addcontentsline{toc}{section}{\textbf{Исходные данные для выполнения расчетной работы}}

Задан непрерывный объект управления (НОУ) с помощью передаточной функции (ПФ) «вход-выход (ВВ)»
\begin{equation}\label{eq_pf0}
	\Phi (s, q) = \cfrac{b_0 (1 + q_1) s + b_1 (1 + q_2)}{\left[a_0 (1+q_3)s + a_1 (1+q_4)\right]\left[a_2 (1+q_5) s^2 + a_3 (1 + q_6) s + a_4 (1 + q_7)\right]}
\end{equation}
где $q_{10}=q_{20}=q_{30}=q_{40}=q_{50}=q_{60}=q_{70}=0$~--- номинальные значения параметров $q_{j0}, j = \overline{1,7}$.

Необходимо проделать работу в соответствии с заданием на расчетно-исследовательскую  работу магистранта (РИРМ). Исходные данные для варианта~№17 БББААААА указаны в таблице~\ref{problem_data}.


	\begin{table}[h!]
	\caption{Исходные данные}
	\begin{tabular}{|p{0.5\linewidth}|p{0.4\linewidth}|}
%\hline
%Параметр & Значения \\
\hline
1.1. Значения параметров ПФ & 
$b_0 = 0; b_1 = 0.67; a_0=0; a_1 = 1; a_2 = 16; a_3 = 3; a_4 = 10$
\\
\hline
1.2. Базис описания НОУ & канонический наблюдаемый
\\
\hline
2.1. Интервал дискретности & $\Delta{t} = 0.03$с\\
\hline
2.2. Метод перехода к ДОУ & заменой производной отношением конечных малых
\\
\hline
3. Характеристическая частота  & $\omega_0 = 3 c^{-1}$\\
\hline
5. Граничные (угловые) значения параметра $q_j$  & $\underline{q_j} = -0.2; \overline{q_j} = 0.2$ \\ 
\hline
6. Относительная интервальность матрицы состояния системы & $\delta_{IR} F = 0.02$\\
\hline
7. Величина параметрической неопределенности  & $\underline{q_j} = -0.2; \overline{q_j} = 0.2$\\
\hline
	\end{tabular}
	\label{problem_data}
\end{table}



\newpage