\section{Построение модели траекторной чувствительности непрерывного объекта управления и результаты ее исследования}\label{problem_1}

\subsection{Непрерывный объект управления в форме вход-состояние-выход}

Передаточная функция заданного объекта управления имеет следующий вид:
\begin{equation}\label{eq_OU}
\Phi (s, q) = \cfrac{0.67(1+q_{2})} {(1+q_{4})(16(1+q_{5})s^{2}+3(1+q_{6})s+10(1+q_{7}))}.
\end{equation}

Для составления векторно-матричного описания ОУ запишем ПФ в форме
\begin{equation*}\label{eq_pf_coc}
\Phi (s, q) = \cfrac{\cfrac{0.67(1+q_{2})} {16(1+q_{5})(1+q_{4})}  } {s^2 + \cfrac{3(1+q_{6})} {16(1+q_{5})} s + \cfrac{5(1+q_{7})} {8(1+q_{5})}}.
\end{equation*}

В каноническом управляемом базисе, векторно-матричное представление объекта управления  имеет следующий вид:
\begin{equation}\label{eq_iso_coc}
\begin{cases}
\dot x(t,q) &= A(q) x(t,q) + B u(t)\\
y(t,q) &= C(q) x(t,q)
\end{cases},
\end{equation}
где
\begin{equation}
	A(q) =
	\begin{bmatrix}
		0 & -\cfrac{5(1+q_{7})} {8(1+q_{5})}\\
	1 & - \cfrac{3(1+q_{6})}  {16(1+q_{5})}
	\end{bmatrix},
\end{equation}
\begin{equation}
	B =
	\begin{bmatrix}
		\cfrac{0.67(1+q_{2})}  {16(1+q_{5})(1+q_{4})}\\
		0
	\end{bmatrix},
\end{equation}

\begin{equation}
	C(q) =
	\begin{bmatrix}
	0& 1
	\end{bmatrix}.
\end{equation}

\subsection{Модель траекторной чувствительности непрерывного объекта управления}

Передаточная функция номинального объекта управления при $ q_{1_{0}}=...=q_{7_{0}}=0 $ имеет следующий вид:
\begin{equation}
	\Phi(s, 0) = \cfrac{\cfrac{0.67}{16}}{s^2 + \cfrac{3}{16} s + \cfrac{5}{8}}. 
\end{equation}

Матрицы модели вход-состояние-выход номинального объекта управления имеют следующие реализации:
\begin{align*}
A =
\begin{bmatrix}
0 & -\cfrac{5} {8}\\
1 & - \cfrac{3}{16}
\end{bmatrix};
B =
\begin{bmatrix}
\cfrac{0.67}{16}\\
0
\end{bmatrix};
C =
\begin{bmatrix}
0 & 1
\end{bmatrix}.
\end{align*}

Введем обозначения
\begin{align*}
	&A_{q_j} = \cfrac{\partial{A(q)}}{\partial{q_j}} \bigg|_{q=q_0},
	B_{q_j} = \cfrac{\partial{B(q)}}{\partial{q_j}} \bigg|_{q=q_0},
	C_{q_j} = \cfrac{\partial{C(q)}}{\partial{q_j}} \bigg|_{q=q_0},\\
	&A(q)|_{q=q_0} = A,
	B(q)|_{q=q_0} = B,
	C(q)|_{q=q_0} = C,\\
	&x(t,q)|_{q=q_0} = x(t),
	y(t,q)|_{q=q_0} = y(t),\\
	&\cfrac{\partial{x(t,q)}}{\partial{q_j}} \bigg|_{q=q_0} = \sigma_j(t),
	\cfrac{\partial{y(t,q)}}{\partial{q_j}} \bigg|_{q=q_0} = \eta_j(t)
\end{align*}

Теперь для $j$-й модели траекторной чувствительности получим представление модели траекторной чувствительности:
\begin{equation}\label{eq_mts}
	\begin{cases}
		\dot \sigma_j(t) &= A \sigma_j(t) + A_{q_j} x(t) + B_{q_j} u(t); 
		\sigma_j (0) = 0\\
		\eta_j (t) &= C \sigma_j (t) + C_{q_j} x(t)
	\end{cases}
\end{equation}

Модель траекторной чувствительности будет генерировать функции траекторной чувствительности $\sigma_j (t)$ по состоянию и $\eta_j (t)$ по выходу, если ее дополнить моделью номинального объекта управления~\ref{eq_iso_coc}.

На состояние заданного объекта управления влияют $p = 5$ (далее, под записью $j = \overline{1, p}$ будет подразумеваться, что $j = 1,2,3,4,6,7$) параметров: $q_2, q_4, q_5, q_6, q_7$. Вычислим матрицы моделей траекторной чувствительности используя выше введенные обозначения:
\begin{align}
	&A_{q_2} = 
	\begin{bmatrix}
		0 & 0\\
		0 & 0
	\end{bmatrix};
	B_{q_2} = 
	\begin{bmatrix}
		\cfrac{0.67}{16}\\
		0
	\end{bmatrix};
	C_{q_2} = 
	\begin{bmatrix}
		0 & 0
	\end{bmatrix};\\
	&A_{q_4} = 
	\begin{bmatrix}
	0 & 0\\
	0 & 0
	\end{bmatrix};
	B_{q_4} = 
	\begin{bmatrix}
	-\cfrac{0.67}{16}\\
	0
	\end{bmatrix};
	C_{q_4} = 
	\begin{bmatrix}
	0 & 0
	\end{bmatrix};\\
	&A_{q_5} = 
	\begin{bmatrix}
	0 & \cfrac{5}{8}\\
	0 & \cfrac{3}{16}
	\end{bmatrix};
	B_{q_5} = 
	\begin{bmatrix}
	-\cfrac{0.67}{16}\\
	0
	\end{bmatrix};
	C_{q_5} = 
	\begin{bmatrix}
	0 & 0
	\end{bmatrix};\\
	&A_{q_6} = 
	\begin{bmatrix}
	0 & 0\\
	0 & - \cfrac{3}{16}
	\end{bmatrix};
	B_{q_6} = 
	\begin{bmatrix}
	0\\
	0
	\end{bmatrix};
	C_{q_6} = 
	\begin{bmatrix}
	0 & 0
	\end{bmatrix};\\
	&A_{q_7} = 
	\begin{bmatrix}
	0 & -\cfrac{5}{8}\\
	0 & 0
	\end{bmatrix};
	B_{q_7} = 
	\begin{bmatrix}
	0\\
	0
	\end{bmatrix};
	C_{q_7} = 
	\begin{bmatrix}
	0 & 0
	\end{bmatrix};
\end{align}

\subsection{Ранжирование параметров}

Оценка управляемости системы, состоящей из моделей номинальной и траекторной чувствительности параметром$ q_{j} $:

\begin{equation}
\tilde{x}_j=
\begin{bmatrix}
x \\ \sigma_j
\end{bmatrix}, dim(\tilde{x})=2n, \dot{\tilde{x}_j}(t)=\tilde{A}_j\tilde{x}_j(t)+\tilde{B}_j u(t), \tilde{x}_j(0)=
\begin{bmatrix}
x(0)\\0
\end{bmatrix}, 
\end{equation}

\begin{equation}
x(t)=\tilde{C}_{xj}\tilde{x}_j(t), \sigma_j(t)=\tilde{C}_{\sigma j}\tilde{x}_j(t), \eta_j(t)=\tilde{C}_{\eta j}\tilde{x}(t).
\end{equation}
где

\begin{align*}
\tilde{A}_j=
\begin{bmatrix}
A&0\\
A_{qj}&A
\end{bmatrix},
\tilde{B}_j=
\begin{bmatrix}
B\\B_{qj}
\end{bmatrix},
\tilde{C}_{xj}=
\begin{bmatrix}
I_{n\times n}&0_{n\times n}
\end{bmatrix},\\
\tilde{C}_{\sigma j}=
\begin{bmatrix}
0_{n\times n}&I_{n\times n}
\end{bmatrix},
\tilde{C}_{\eta j}=
\begin{bmatrix}
C_{qj} & C
\end{bmatrix}
\end{align*}

\begin{align*}
\tilde{A}_{2,4}=
\begin{bmatrix}
0&-\cfrac{5}{18}&0&0\\
1&-\cfrac{3}{16}&0&0\\
0&0&0&-\cfrac{5}{8}\\
0 &0&1&-\cfrac{3}{16}
\end{bmatrix},
\tilde{A}_5=
\begin{bmatrix}
0&-\cfrac{5}{8}&0&0\\
1&-\cfrac{3}{16}&0&0\\
0&\cfrac{5}{8}&0&-\cfrac{5}{8}\\
0&\cfrac{3}{16}&1&-\cfrac{3}{16}
\end{bmatrix},
\tilde{A}_6=
\begin{bmatrix}
0&-\cfrac{5}{8}&0&0\\
1&-\cfrac{3}{16}&0&0\\
0&0&0&-\cfrac{5}{8}\\
0&-\cfrac{3}{16}&1&-\cfrac{3}{16}
\end{bmatrix},
\end{align*}

\begin{align*}
\tilde{A}_7=
\begin{bmatrix}
0&-\cfrac{5}{8}&0&0\\
1&-\cfrac{3}{16}&0&0\\
0&-\cfrac{5}{8}&0&-\cfrac{5}{8}\\
0&0&1&-\cfrac{3}{16}
\end{bmatrix},
\tilde{B}_2=
\begin{bmatrix}
\cfrac{0.67}{16}\\
0\\
\cfrac{0.67}{16}\\
0
\end{bmatrix},
\tilde{B}_{4,5}=
\begin{bmatrix}
\cfrac{0.67}{16}\\0\\
-\cfrac{0.67}{16}\\
0
\end{bmatrix},
\tilde{B}_{6,7}=
\begin{bmatrix}
\cfrac{0.67}{16}\\
0\\
0\\
0
\end{bmatrix},
\end{align*}

\begin{align*}
\tilde{C}_x_{2,4,5,6,7}=
\begin{bmatrix}
1&0&0&0\\0&1&0&0
\end{bmatrix},
\tilde{C}_{2,4,5,6}=
\begin{bmatrix}
1&0&0&0
\end{bmatrix},
\tilde{C}_{\sigma_{2,4,5,6,7}}=
\begin{bmatrix}
0&0&1&0\\0&0&0&1
\end{bmatrix},
\end{align*}

\begin{align*}
\tilde{C}_{\eta_{2,4,5,6,7}}=
\begin{bmatrix}
0&0&0&1
\end{bmatrix}
\end{align*}

Требования к ресурсам управления заметно снижаются, если изначально ограничиться задачей обеспечения траекторной нечувствительности выхода проектируемой системы. На уровне требований к структурным свойствам агрегированной системы задача сводится к контролю управляемости тройки матриц $(\tilde{C}_{\eta_j}, \tilde{A}_{j}, \tilde{B}_{j})$ и количественной оценке эффекта управления по переменной $\eta_j$ при приложении управления $u(t)$ фиксированной нормы с помощью сингулярных чисел матрицы управляемости

Для оценки управляемости по выходу проверим матрицы $ \tilde{C}_{\eta j},\tilde{A}_j,\tilde{B}_j $:

\begin{equation}
	\tilde{W}_{y \eta_j} =
	\begin{bmatrix}
		\tilde{C}_{\eta_j} \tilde{B}_{j} &
		\tilde{C}_{\eta_j} \tilde{A}_{j} \tilde{B}_{j} &
		\tilde{C}_{\eta_j} \tilde{A}_{j}^2 \tilde{B}_{j} &		
		\cdots &
		\tilde{C}_{\eta_j} \tilde{A}_{j}^{2n-1} \tilde{B}_{j}
	\end{bmatrix}
\end{equation}

Рассчитаем матрицы управляемости $\tilde{W}_{\eta_j}$
\begin{align*}
	&\tilde{W}_{y \eta_2} =
	\begin{bmatrix}
		0 &  0.041875 &   −0.0078516& −0.0246997  	
	\end{bmatrix}, 
	\\
	&\tilde{W}_{y \eta_4} =
	\begin{bmatrix}
		0   &   −0.041875 & 0.0078516 &   0.0246997  
	\end{bmatrix}, 
	\\
	&\tilde{W}_{y \eta_5} =
	\begin{bmatrix}
		0 & −0.041875    &    0.0157031 & 0.0479272 
	\end{bmatrix}, 
	\\
	&\tilde{W}_{y \eta_6} =
	\begin{bmatrix}
		0   &   0   &   −0.0078516  &  0.0029443 
	\end{bmatrix}, 
	\\
	&\tilde{W}_{y \eta_7} =
	\begin{bmatrix}
		0 & 0 &   0 & −0.0261719
	\end{bmatrix}.
\end{align*}

Вычислим для полученных матриц управляемости сингулярные числа
\begin{align}\label{singul_nums}
	&\alpha\{\tilde{W}_{y \eta_{2}}\} = 0.0492467,
	\alpha\{\tilde{W}_{y \eta_{4}}\} = 0.0492467,\\	
	&\alpha\{\tilde{W}_{y \eta_{5}}\} = 0.0655525,	
	\alpha\{\tilde{W}_{y \eta_{6}}\} = 0.0083855,\\		
	&\alpha\{\tilde{W}_{y \eta_{7}}\} = 0.0261719.
		\label{singul_nums_end}
\end{align}


Ранжирование параметров $q_j$ осуществляется по значению сингулярных чисел матриц управляемости. 
Чем эти числа меньше, тем большими по норме управлениями достигается асимптотическая траекторная нечувствительность компонента yj(t) вектора выхода y(t). Отсюда следует, что асимптотическая сходимость к нулю дополнительного движения будет требовать все меньшего количества затрат при следующем расположении qj : q6, q7, q2, q4, q5.



\newpage